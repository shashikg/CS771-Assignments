\documentclass[a4paper,11pt]{article}

\usepackage{mlsubmit}

\begin{document}

\initmlsubmision{1} % assignment number
{Shashi Kant Gupta}   % your name
{160645}	% your roll number

\begin{mlsolution}

  \begin{enumerate}
    \item
      \textbf{For Tree A:} \\
        From leaf node 1 min. number of element in a class is 100 and from leaf node 2 also min. number of element in a class is 100. Therefore,
        \begin{equation}
          MissclassificationRate = \frac{100 + 100}{800} = \frac{1}{4}
        \end{equation}
      \textbf{For Tree B:} \\
        From leaf node 1 min. number of element in a class is 200 and from leaf node 2 min. number of element in a class is 0. Therefore,
        \begin{equation}
          MissclassificationRate = \frac{200 + 0}{800} = \frac{1}{4}
        \end{equation}
        Yes, they are equal!
    \item
      \textbf{For Tree A:}
      {\begin{equation}
        H(S) = -\frac{1}{2}(log(\frac{1}{2}) + log(\frac{1}{2}))
             = 1
      \end{equation}}
      \begin{equation}
        H(S_{1}) = -\frac{1}{4}(log(\frac{1}{4}) + 3log(\frac{3}{4}))
                 = 0.811
      \end{equation}
      \begin{equation}
        H(S_{2}) = -\frac{1}{4}(log(\frac{1}{4}) + 3log(\frac{3}{4}))
                 = 0.811
      \end{equation}
      \begin{equation}
        \Rightarrow IG_{A} = 1 - \frac{1}{2}0.811 - \frac{1}{2}0.811
                       = 0.189
      \end{equation}
      \textbf{For Tree B:}
      {\begin{equation}
        H(S) = -\frac{1}{2}(log(\frac{1}{2}) + log(\frac{1}{2}))
             = 1
      \end{equation}}
      \begin{equation}
        H(S_{1}) = -\frac{1}{3}(log(\frac{1}{3}) + 2log(\frac{2}{3}))
                 = 0.918
      \end{equation}
      \begin{equation}
        H(S_{2}) = -\frac{2}{2}log(\frac{2}{2})
                 = 0.
      \end{equation}
      \begin{equation}
        \Rightarrow IG_{B} = 1 - \frac{3}{4}0.918
                       = 0.311
      \end{equation}
      Since $IG_{B} > IG_{A}$. Therefore, Tree B is better decision tree!

    \item
      From (1) its not possible to predict which tree to select but from (2) we can decide Tree B to better. This makes sense because entropy is better deciding factor for decision tree.
  \end{enumerate}}

\end{mlsolution}

\begin{mlsolution}

\noindent Yes, \textbf{1NN} will be consistent in this case!\\
Since, there are infinite numbers of training data and each of them are correctly labeled without any noise. That means whenever you get a test data you can always find
a training data completing close to it probability of finding such point will tend to 1 when the numbers of train data goes to infinity!
 Therefore, you can always classify them with no error in classifcation.
 In simple words you already have that test data in your training set!



\end{mlsolution}

\begin{mlsolution}

Given: $\hat\vw = (\vX^{T}\vX)^{-1}\vX^{T}\vy$. Therefore, a prediction at test input $\vx_{*}$ can be written as $y_{*} = \hat\vw^{T}\vx_{*} = \vx_{*}^{T}\hat\vw$ \\
Therefore,
\begin{equation}
  y_{*} = \vx_{*}^{T}(\vX^{T}\vX)^{-1}\vX^{T}\vy
\end{equation}
\begin{equation}
  \Rightarrow y_{*} = \vW\vy
\end{equation}
where, $\vW = \vx_{*}^{T}(\vX^{T}\vX)^{-1}\vX^{T}$. Therefore, $\vW = (w_{1} w_{2} . . . w_{N})$ comes out to be a 1xN matrix. We can also write $\vy = (y_{1} y_{2} . . . y_{N})^{T}$.
\\ Therefore,
\begin{equation}
  y_{*} = \vW\vy = \sum_{n=1}^Nw_{n}y_{n}
\end{equation}
Therefore, $w_{n}$ are the $n^{th}$ index of the 1xN matrix $\vW$. Knowing that $\vX$ is matrix whose rows are the N training vectors $\vx_{n}$, $w_{n}$ can be written as:
\begin{equation}
  w_{n} = \vx_{*}^T(\vX^{T}\vX)^{-1}\vx_{n}
\end{equation}
So, $w_{n}$ depends on the input $\vx_{*}$ and all the training data's from $\vx_{1}$ to $\vx_{n}$. Since, there exist $X^{T}X$ term in the expressin of $w_{n}$.
Which is not in case with weighted kNN, where individual weights depends only on $\vx_{*}$ and $\vx_{n}$. Also, $\vx_{*}$ comes in numerator in case of this while for kNN it comes in denominator.
Another difference is that $w_{n}$ are expressed as products of $\vx_{*}$ while in kNN they are expressed as sum in denominator.

\end{mlsolution}

\begin{mlsolution}

We can use a diagonal matrix $\vM = diag(m_{11} m_{22} . . . m_{DD})$ to define a new regularization term as $\vw_{T}\vM\vw$, inspired from mahalanobis distance!\\
Therefore, our loss becomes:
\begin{equation}
  \vL(\vw) = \sum_{n=1}^{N}(y_{n} - \vw^{T}\vx_{n})^2 + \vw_{T}\vM\vw
\end{equation}
We can also write it as,
\begin{equation}
  \vL(\vw) = (\vy - \vX\vw)^{T}(\vy - \vX\vw) + \vw_{T}\vM\vw
\end{equation}
\begin{equation}
  \Rightarrow \frac{\partial\vL(\vw)}{\partial\vw} = -2\vX^{T}(\vy - \vX\vw ) + 2\vM\vw
\end{equation}
For closed form solution put $\frac{\partial\vL(\vw)}{\partial\vw} = 0$. Therefore,
\begin{equation}
  \vX^{T}\vy = \vX^{T}\vX\vw + \vM\vw
\end{equation}
\begin{equation}
  \Rightarrow \vX^{T}\vy = (\vX^{T}\vX + \vM)\vw
\end{equation}
\begin{equation}
  Therefore, \vw = (\vX^{T}\vX + \vM)^{-1}\vX^{T}\vy
\end{equation}

\end{mlsolution}

\begin{mlsolution}

  \begin{equation}
    \vL(\vW) = TRACE[(\vY - \vX\vB\vS)^{T}(\vY - \vX\vB\vS)]
  \end{equation}
  \begin{equation}
     = TRACE[(\vY^{T} - \vS^{T}\vB^{T}\vX^{T})(\vY - \vX\vB\vS)]
  \end{equation}
  \begin{equation}
    \Rightarrow \vL(\vW) = TRACE[\vY^{T}\vY - \vY^{T}\vX\vB\vS - \vS^{T}\vB^{T}\vX^{T}\vY + \vS^{T}\vB^{T}\vX^{T}\vX\vB\vS]
  \end{equation}
  Using the identities we get,
  \begin{equation}
    \Rightarrow \frac{\partial\vL(\vW)}{\partial\vS} = -(\vY^{T}\vX\vB)^{T} - \vB^{T}\vX^{T}\vY + ((\vB^{T}\vX^{T}\vX\vB) + (\vB^{T}\vX^{T}\vX\vB)^{T})\vS
  \end{equation}
  For closed form solution keeping $\vB$ constant put $\frac{\partial\vL(\vW)}{\partial\vS} = 0$. Therefore,
  \begin{equation}
    \vB^{T}\vX^{T}\vY = (\vB^{T}\vX^{T}\vX\vB)\vS
  \end{equation}
  \begin{equation}
    \Rightarrow \vS = (\vB^{T}\vX^{T}\vX\vB)^{-1}\vB^{T}\vX^{T}\vY
  \end{equation}
  writing $\vB^{T}\vX^{T} = (\vX\vB)^{T}$,
  \begin{equation}
    \Rightarrow \vS = ((\vX\vB)^{T}\vX\vB)^{-1}(\vX\vB)^{T}\vY
  \end{equation}
  replacing $\vX\vB = \vV$, we get,
  \begin{equation}
    \Rightarrow \vS = (\vV^{T}\vV)^{-1}\vV^{T}\vY
  \end{equation}
  Therefore, from the above equation when putting $\vX\vB = \vV$ we can see that the solution is identifical to the solution of
standard multi-output regression where $\vX$ have been transformed to $\vV$.


\end{mlsolution}

\begin{mlsolution}
\noindent\textbf{Method 1:} \\
Test accuracy with Euclidean distance is 46.8932038835 \\
I had also tried implementing the mahalanobis distance but not much improvement observed. Following are the results based on number of iterations I runned to optimise theta: \\
Test accuracy for iter = 5 is: 47.1844660194 \\
Test accuracy for iter = 10 is: 48.2362459547\\
Test accuracy for iter = 15 is: 48.8996763754\\
Test accuracy for iter = 20 is: 49.7249190939\\
Test accuracy for iter = 25 is: 50.1294498382\\
Test accuracy for iter = 30 is: 50.2103559871\\

\noindent\textbf{Method 2:} \\
Test accuracy for $\lambda$ = 0.01 is: 58.0906148867 \\
Test accuracy for $\lambda$ = 0.1 is: 59.5469255663 \\
Test accuracy for $\lambda$ = 1 is: 67.3948220065 \\
Test accuracy for $\lambda$ = 10 is: 73.284789644 \\
Test accuracy for $\lambda$ = 20 is: 71.6828478964 \\
Test accuracy for $\lambda$ = 50 is: 65.0809061489 \\
Test accuracy for $\lambda$ = 100 is: 56.4724919094 \\
$\lambda$ = 10 gives the best result!

\end{mlsolution}


\end{document}
